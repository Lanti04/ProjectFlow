% ProjectFlow Learning Portfolio
% Author: Taulant Llumnica
% Date: November 2025 – Ongoing

\documentclass[12pt, a4paper]{report}

% ========== PACKAGES ==========
\usepackage[utf-8]{inputenc}
\usepackage[english]{babel}
\usepackage[margin=1in]{geometry}
\usepackage{graphicx}
\usepackage{hyperref}
\usepackage{fancyhdr}
\usepackage{setspace}
\usepackage{booktabs}
\usepackage{array}
\usepackage{xcolor}
\usepackage{listings}
\usepackage{tcolorbox}

% ========== FORMATTING ==========
\onehalfspacing
\hypersetup{colorlinks=true, linkcolor=blue, urlcolor=blue}
\pagestyle{fancy}
\fancyhf{}
\fancyhead[R]{\thepage}
\fancyfoot[C]{ProjectFlow Learning Portfolio}

% ========== COLOR DEFINITIONS ==========
\definecolor{darkblue}{HTML}{1e3a8a}
\definecolor{lightgray}{HTML}{f3f4f6}

% ========== CUSTOM COMMANDS ==========
\newcommand{\sectionline}{{\color{darkblue}\hrule height 1pt}}

% ========== DOCUMENT START ==========
\begin{document}

% ========== TITLE PAGE ==========
\begin{titlepage}
\centering
\vspace*{2cm}

{\Huge \textbf{ProjectFlow}}
\vspace{0.5cm}

{\Large \textit{A Full-Stack Project Management Application}}
\vspace{2cm}

\sectionline
\vspace{1cm}

{\large \textbf{Learning Portfolio}}
\vspace{1.5cm}

{\large \textbf{Author:} Taulant Llumnica}
\vspace{0.3cm}

{\large \textbf{Date Range:} November 2025 – Ongoing}
\vspace{2cm}

\vfill

{\normalsize
\textit{This document reflects my learning journey and technical development through the design and implementation of a full-stack web application.}
}

\end{titlepage}

% ========== TABLE OF CONTENTS ==========
\newpage
\tableofcontents
\newpage

% ========== ABSTRACT ==========
\chapter*{Abstract}
\addcontentsline{toc}{chapter}{Abstract}

This portfolio documents my journey in developing \textbf{ProjectFlow}, a comprehensive full-stack web application designed to help users manage projects and tasks efficiently. The project spans multiple technological domains including frontend development (React + Vite), backend architecture (Node.js + Express), and database design (PostgreSQL).

\textbf{Key Learning Outcomes:}
\begin{itemize}
    \item Full-stack web application development from design to deployment
    \item User authentication and security best practices with JWT tokens
    \item RESTful API design and implementation
    \item Database schema design and optimization
    \item State management and component architecture in modern frontend frameworks
    \item Real-time user feedback and interactive UI design
    \item Documentation and technical communication skills
\end{itemize}

This portfolio serves as evidence of both technical competence and reflective learning throughout the project lifecycle.

\newpage

% ========== PART I: PERSONAL LEARNING PLAN ==========
\part{Personal Learning Plan}

% ========== CHAPTER 1: STUDENT PROFILE ==========
\chapter{Student Profile}

\section{Academic Background}
I am a software development student with a strong foundation in computer science fundamentals. My studies have provided me with:
\begin{itemize}
    \item Core programming knowledge (JavaScript, Python, SQL)
    \item Software engineering principles and best practices
    \item Database design and management concepts
    \item Web development fundamentals (HTML, CSS, JavaScript)
\end{itemize}

\section{Technical Strengths}
\begin{itemize}
    \item \textbf{Frontend Development:} Proficient in React and modern JavaScript/ES6+
    \item \textbf{Backend Development:} Experience with Node.js and Express.js
    \item \textbf{Database Design:} SQL and relational database modeling
    \item \textbf{Version Control:} Git and GitHub workflows
    \item \textbf{CSS Styling:} Tailwind CSS and responsive design principles
\end{itemize}

\section{Areas of Development}
\begin{itemize}
    \item \textbf{Advanced API Security:} Moving beyond basic JWT authentication
    \item \textbf{Deployment \& DevOps:} Docker containerization and production environments
    \item \textbf{Performance Optimization:} Advanced caching and database optimization
    \item \textbf{Testing Methodologies:} Unit testing, integration testing, and test-driven development
    \item \textbf{Technical Documentation:} Creating clear, maintainable documentation
\end{itemize}

% ========== CHAPTER 2: REFLECTION AND INITIAL LEARNING OBJECTIVES ==========
\chapter{Reflection and Initial Learning Objectives}

\section{Initial Reflection}

Before embarking on the ProjectFlow project, I recognized a gap in my practical experience with full-stack development. While my coursework provided theoretical knowledge, I needed a real-world project to consolidate learning and understand how different components work together in a production environment.

I wanted to move beyond isolated exercises and create something meaningful that combines multiple technologies in a coherent architecture.

\section{Learning Objectives}

\subsection{Primary Objectives}
\begin{enumerate}
    \item Design and implement a complete full-stack web application
    \item Master JWT-based authentication and user session management
    \item Create a RESTful API with proper endpoint design and error handling
    \item Optimize a PostgreSQL database with indexes and efficient queries
    \item Build an interactive React frontend with real-time user feedback
\end{enumerate}

\subsection{Secondary Objectives}
\begin{enumerate}
    \item Practice code organization and maintainability
    \item Implement responsive design principles for multiple devices
    \item Document code and create comprehensive README files
    \item Use Docker for containerization and local development
    \item Version control and collaborative development practices
\end{enumerate}

\section{Expected Outcomes}

By the end of this project, I aim to:
\begin{itemize}
    \item Have a fully functional, deployable web application
    \item Demonstrate competency across the entire development stack
    \item Understand real-world architectural decisions and trade-offs
    \item Communicate technical concepts clearly through documentation
    \item Have a portfolio piece showcasing my capabilities to potential employers
\end{itemize}

% ========== CHAPTER 3: 360-DEGREE FEEDBACK ==========
\chapter{360-Degree Feedback}

\section{Feedback Summary}

This section can be populated with feedback from:
\begin{itemize}
    \item \textbf{Instructor/Mentor Feedback:} Comments on technical implementation and code quality
    \item \textbf{Peer Review:} Insights from colleagues or study partners
    \item \textbf{Self-Assessment:} Personal evaluation of strengths and areas for improvement
    \item \textbf{User Testing:} Feedback from actual users testing the application
\end{itemize}

\section{Planned Feedback Sessions}

\begin{tabular}{|l|l|l|}
\hline
\textbf{Feedback Source} & \textbf{Date} & \textbf{Focus Area} \\
\hline
Self-Assessment & December 2025 & Architecture \& Code Quality \\
\hline
Code Review & December 2025 & Best Practices \& Security \\
\hline
User Testing & January 2026 & UX \& Functionality \\
\hline
Mentor Feedback & January 2026 & Overall Development \\
\hline
\end{tabular}

% ========== CHAPTER 4: FURTHER REFLECTION ==========
\chapter{Further Reflection and Lessons Learned}

\section{Key Insights Gained}

\subsection{Technical Insights}
\begin{itemize}
    \item Understanding the importance of database design on application performance
    \item Real-world complexity of authentication and security implementation
    \item The value of proper error handling and user feedback mechanisms
    \item How frontend and backend must be designed with each other in mind
\end{itemize}

\subsection{Process Insights}
\begin{itemize}
    \item Iterative development allows for continuous improvement and refinement
    \item Documentation should be written alongside code, not after
    \item Testing early prevents cascading issues later in development
    \item Clear project structure makes scaling and maintenance easier
\end{itemize}

\section{Challenges Overcome}

\begin{description}
    \item[Database Connection Management] Initially struggled with proper connection pooling; resolved by implementing a centralized pool
    \item[State Management Complexity] Managing state across multiple components; solved with clear component hierarchy
    \item[API Error Handling] Inconsistent error responses; standardized with common error handling middleware
    \item[Password Security] Implemented bcryptjs for proper password hashing and comparison
\end{description}

\section{Skills Development}

\subsection{Technical Skills}
\begin{itemize}
    \item Deepened understanding of HTTP/REST architecture
    \item Improved SQL query optimization skills
    \item Enhanced React component design patterns
    \item Gained practical experience with JWT and encryption
\end{itemize}

\subsection{Soft Skills}
\begin{itemize}
    \item Improved technical documentation and communication
    \item Enhanced problem-solving and debugging abilities
    \item Better project planning and time management
    \item Increased confidence in decision-making
\end{itemize}

\newpage

% ========== PART II: THEORETICAL LEARNING PLAN ==========
\part{Theoretical Learning Plan}

% ========== CHAPTER 5: CONTEXT AND RESEARCH QUESTION ==========
\chapter{Context and Research Question}

\section{Project Context}

Modern software development requires developers to understand and work across multiple layers of a full-stack application. Project management is a universal need, making it an ideal domain for learning full-stack development concepts.

\section{Main Research Questions}

\begin{enumerate}
    \item \textbf{How can I design a scalable, secure full-stack application?}
    \begin{itemize}
        \item Exploring RESTful architecture
        \item Implementing proper authentication and authorization
        \item Designing efficient database schemas
    \end{itemize}
    
    \item \textbf{What best practices should guide frontend and backend development?}
    \begin{itemize}
        \item Component organization and reusability
        \item API endpoint design and documentation
        \item Error handling and user feedback
    \end{itemize}
    
    \item \textbf{How can I create a professional, maintainable codebase?}
    \begin{itemize}
        \item Code organization and naming conventions
        \item Documentation and comments
        \item Version control and collaborative practices
    \end{itemize}
\end{enumerate}

\section{Relevance to Industry}

ProjectFlow demonstrates skills directly applicable to professional development:
\begin{itemize}
    \item Full-stack understanding required by many employers
    \item CRUD operations are fundamental to most business applications
    \item User authentication is crucial for real-world applications
    \item Database design reflects real-world complexity
\end{itemize}

% ========== CHAPTER 6: LEARNING STRATEGY ==========
\chapter{Learning Strategy}

\section{Development Methodology}

\begin{itemize}
    \item \textbf{Iterative Development:} Build features incrementally with testing at each stage
    \item \textbf{Test-Driven Thinking:} Consider error cases and edge cases during implementation
    \item \textbf{Documentation-First:} Plan and document before coding
    \item \textbf{Code Review:} Regularly review own code for quality and improvements
\end{itemize}

\section{Learning Resources}

\begin{itemize}
    \item \textbf{Official Documentation:} React, Express.js, PostgreSQL docs
    \item \textbf{Best Practices Guides:} OWASP for security, REST API design patterns
    \item \textbf{Practical Examples:} GitHub repositories and code samples
    \item \textbf{Community Resources:} Stack Overflow, forums, and technical blogs
\end{itemize}

\section{Implementation Approach}

\begin{enumerate}
    \item \textbf{Planning Phase:} Sketch architecture, database schema, and user flows
    \item \textbf{Backend Development:} Build API with proper structure and error handling
    \item \textbf{Frontend Development:} Create UI components and integrate with backend
    \item \textbf{Testing \& Refinement:} Test functionality, fix bugs, optimize performance
    \item \textbf{Documentation:} Document code, API, and usage
    \item \textbf{Deployment:} Containerize and prepare for production deployment
\end{enumerate}

% ========== CHAPTER 7: GOALS AND DELIVERABLES ==========
\chapter{Goals and Deliverables}

\section{Alignment of Goals and Deliverables}

\begin{center}
\begin{tabular}{|p{3cm}|p{5cm}|}
\hline
\textbf{Learning Goal} & \textbf{Deliverable} \\
\hline
Full-Stack Development & Functional ProjectFlow Application \\
\hline
API Design & Documented RESTful API Endpoints \\
\hline
Database Design & Normalized PostgreSQL Schema with Indexes \\
\hline
Frontend Skills & Responsive React Interface with Tailwind CSS \\
\hline
Authentication & JWT-based User Authentication System \\
\hline
Code Quality & Well-Organized, Commented Source Code \\
\hline
Documentation & Comprehensive README and API Documentation \\
\hline
DevOps Basics & Docker Configuration for Local Development \\
\hline
\end{tabular}
\end{center}

\section{Success Criteria}

\begin{itemize}
    \item \textbf{Functionality:} All core features work as designed
    \item \textbf{Security:} Authentication and data validation implemented correctly
    \item \textbf{Performance:} Application loads quickly with optimized queries
    \item \textbf{Usability:} Intuitive interface with clear user feedback
    \item \textbf{Code Quality:} Clean, well-organized, maintainable code
    \item \textbf{Documentation:} Complete and understandable documentation
\end{itemize}

% ========== CHAPTER 8: BREAKDOWN OF ACTIVITIES (TIMELINE) ==========
\chapter{Breakdown of Activities and Timeline}

\section{Project Timeline}

\subsection{Week 1-2: Planning and Setup (November 2025)}

\begin{itemize}
    \item Define project scope and requirements
    \item Design database schema and create ERD
    \item Plan API endpoint structure
    \item Set up development environment (Node.js, PostgreSQL, Docker)
    \item Initialize Git repository
\end{itemize}

\subsection{Week 3-4: Backend Foundation (November 2025)}

\begin{itemize}
    \item Create Express server with middleware
    \item Implement PostgreSQL connection pool
    \item Build authentication controller (register, login)
    \item Create auth middleware for protected routes
    \item Implement basic error handling
\end{itemize}

\subsection{Week 5-6: Backend Features (December 2025)}

\begin{itemize}
    \item Develop dashboard routes with data fetching
    \item Implement project CRUD operations
    \item Implement task CRUD operations
    \item Add progress calculation logic
    \item Create seed data for testing
\end{itemize}

\subsection{Week 7-8: Frontend Foundation (December 2025)}

\begin{itemize}
    \item Set up React project with Vite
    \item Configure Tailwind CSS
    \item Create authentication pages (Login, Register)
    \item Implement token storage and retrieval
    \item Set up routing with React Router
\end{itemize}

\subsection{Week 9-10: Frontend Features (December 2025 - January 2026)}

\begin{itemize}
    \item Build Dashboard component with state management
    \item Create Project and Task display components
    \item Implement modals for creating/editing content
    \item Add celebration effects (confetti, sounds)
    \item Implement Focus Mode
    \item Build Navbar with navigation
\end{itemize}

\subsection{Week 11-12: Testing, Refinement \& Documentation (January 2026)}

\begin{itemize}
    \item Comprehensive manual testing
    \item Bug fixes and performance optimization
    \item Write comprehensive README
    \item Create API documentation
    \item Docker setup and testing
    \item Code review and cleanup
\end{itemize}

\section{Progress Tracking}

\begin{center}
\begin{tabular}{|p{2.5cm}|p{2cm}|p{2cm}|}
\hline
\textbf{Phase} & \textbf{Status} & \textbf{Completion \%} \\
\hline
Planning & Completed & 100\% \\
\hline
Backend Dev & In Progress & 85\% \\
\hline
Frontend Dev & In Progress & 75\% \\
\hline
Testing & Upcoming & 0\% \\
\hline
Documentation & In Progress & 80\% \\
\hline
\end{tabular}
\end{center}

% ========== CHAPTER 9: INDIVIDUAL STUDY AREAS ==========
\chapter{Individual Study Areas}

\section{Technical Topics Mastered}

\subsection{Frontend Development}
\begin{itemize}
    \item React Hooks and functional components
    \item Component state management and props
    \item React Router for client-side routing
    \item Tailwind CSS utility-first styling
    \item API integration with Axios
    \item Form handling and validation
    \item Responsive design principles
\end{itemize}

\subsection{Backend Development}
\begin{itemize}
    \item Express.js server setup and routing
    \item Middleware implementation (CORS, JSON parsing)
    \item RESTful API design principles
    \item Error handling and status codes
    \item Database connection pooling
    \item Parameterized queries for security
\end{itemize}

\subsection{Database Management}
\begin{itemize}
    \item Relational database design
    \item SQL CRUD operations
    \item Foreign keys and referential integrity
    \item Database indexing for optimization
    \item Query optimization techniques
\end{itemize}

\section{Topics Currently Being Studied}

\subsection{Security and Performance}
\begin{itemize}
    \item Advanced authentication strategies (OAuth2, social login)
    \item Rate limiting and DDoS protection
    \item Database query optimization
    \item Caching strategies (Redis)
    \item Security headers and HTTPS
\end{itemize}

\subsection{DevOps and Deployment}
\begin{itemize}
    \item Docker containerization
    \item Environment configuration management
    \item CI/CD pipelines
    \item Cloud deployment (Vercel, Heroku)
    \item Database migration tools
\end{itemize}

\subsection{Advanced Frontend Concepts}
\begin{itemize}
    \item State management libraries (Redux, Context API)
    \item Performance optimization techniques
    \item Progressive Web Apps (PWAs)
    \item Testing frameworks (Jest, React Testing Library)
    \item Component libraries and design systems
\end{itemize}

\section{Soft Skill Development}

\subsection{Communication Skills}
\begin{itemize}
    \item Technical documentation writing
    \item Code commenting and explanation
    \item API documentation clarity
    \item Explaining technical decisions
\end{itemize}

\subsection{Problem-Solving}
\begin{itemize}
    \item Debugging complex issues across layers
    \item Performance bottleneck identification
    \item Security vulnerability assessment
    \item Creative solution finding under constraints
\end{itemize}

\subsection{Professional Practices}
\begin{itemize}
    \item Version control and Git workflows
    \item Code organization and naming conventions
    \item Project planning and estimation
    \item Time management and priority setting
    \item Continuous learning and self-improvement
\end{itemize}

\newpage

% ========== APPENDICES ==========
\appendix

\chapter{Full Database Schema}

\section{Entity Relationship Diagram}

\begin{tcolorbox}[colback=lightgray, colframe=darkblue, width=\textwidth]
\textbf{NOTE:} An ERD diagram can be inserted here using graphicx package.

The database consists of three main entities: Users, Projects, and Tasks with the following relationships:
\begin{itemize}
    \item One User can have many Projects (1:N)
    \item One Project can have many Tasks (1:N)
    \item Deleting a Project cascades deletion to its Tasks
\end{itemize}
\end{tcolorbox}

\section{SQL Schema}

\subsection{Users Table}
\begin{verbatim}
CREATE TABLE users (
    id SERIAL PRIMARY KEY,
    email VARCHAR(255) UNIQUE NOT NULL,
    password_hash VARCHAR(255) NOT NULL,
    name VARCHAR(100) NOT NULL,
    created_at TIMESTAMPTZ DEFAULT NOW()
);
\end{verbatim}

\subsection{Projects Table}
\begin{verbatim}
CREATE TABLE projects (
    id SERIAL PRIMARY KEY,
    user_id INTEGER NOT NULL REFERENCES users(id) 
        ON DELETE CASCADE,
    title VARCHAR(255) NOT NULL,
    description TEXT,
    deadline DATE,
    status VARCHAR(50) DEFAULT 'active',
    progress INTEGER DEFAULT 0 
        CHECK (progress BETWEEN 0 AND 100),
    created_at TIMESTAMPTZ DEFAULT NOW()
);

CREATE INDEX idx_projects_user_id ON projects(user_id);
CREATE INDEX idx_projects_deadline ON projects(deadline);
\end{verbatim}

\subsection{Tasks Table}
\begin{verbatim}
CREATE TABLE tasks (
    id SERIAL PRIMARY KEY,
    project_id INTEGER NOT NULL REFERENCES projects(id) 
        ON DELETE CASCADE,
    title VARCHAR(255) NOT NULL,
    description TEXT,
    due_date DATE,
    status VARCHAR(50) DEFAULT 'todo',
    created_at TIMESTAMPTZ DEFAULT NOW()
);

CREATE INDEX idx_tasks_project_id ON tasks(project_id);
CREATE INDEX idx_tasks_status ON tasks(status);
CREATE INDEX idx_tasks_due_date ON tasks(due_date);
\end{verbatim}

\chapter{API Endpoints Reference}

\section{Authentication Endpoints}

\begin{center}
\begin{tabular}{|l|l|l|l|}
\hline
\textbf{Method} & \textbf{Endpoint} & \textbf{Auth} & \textbf{Description} \\
\hline
POST & /api/auth/register & No & Create new user \\
\hline
POST & /api/auth/login & No & Authenticate user \\
\hline
GET & /api/auth/me & Yes & Get current user \\
\hline
POST & /api/auth/logout & Yes & End session \\
\hline
\end{tabular}
\end{center}

\section{Dashboard Endpoints}

\begin{center}
\begin{tabular}{|l|l|l|l|}
\hline
\textbf{Method} & \textbf{Endpoint} & \textbf{Auth} & \textbf{Description} \\
\hline
GET & /api/dashboard & Yes & Get projects \& tasks \\
\hline
\end{tabular}
\end{center}

\section{Project Endpoints}

\begin{center}
\begin{tabular}{|l|l|l|l|}
\hline
\textbf{Method} & \textbf{Endpoint} & \textbf{Auth} & \textbf{Description} \\
\hline
POST & /api/projects & Yes & Create project \\
\hline
PUT & /api/projects/:id & Yes & Update project \\
\hline
DELETE & /api/projects/:id & Yes & Delete project \\
\hline
\end{tabular}
\end{center}

\section{Task Endpoints}

\begin{center}
\begin{tabular}{|l|l|l|l|}
\hline
\textbf{Method} & \textbf{Endpoint} & \textbf{Auth} & \textbf{Description} \\
\hline
POST & /api/tasks & Yes & Create task \\
\hline
PATCH & /api/tasks/:id/toggle & Yes & Toggle completion \\
\hline
DELETE & /api/tasks/:id & Yes & Delete task \\
\hline
\end{tabular}
\end{center}

\chapter{Architecture Diagram}

\section{System Architecture}

\begin{tcolorbox}[colback=lightgray, colframe=darkblue, width=\textwidth]
\textbf{Application Architecture:}

\begin{verbatim}
┌─────────────────────────────────────────────────────┐
│                  Frontend (React + Vite)            │
│  ┌──────────────────────────────────────────────┐   │
│  │  Pages: Login, Register, Dashboard           │   │
│  │  Components: Navbar, Modal Forms             │   │
│  │  Libraries: Tailwind CSS, Axios, React Router│   │
│  └──────────────────────────────────────────────┘   │
│              ↓ HTTP Requests / JWT ↓                │
├─────────────────────────────────────────────────────┤
│               Backend (Express.js)                   │
│  ┌──────────────────────────────────────────────┐   │
│  │  Routes: auth, dashboard, projects           │   │
│  │  Controllers: authController                 │   │
│  │  Middleware: CORS, JSON, auth verification  │   │
│  │  Error Handling: standardized responses      │   │
│  └──────────────────────────────────────────────┘   │
│              ↓ SQL Queries / Pool ↓                 │
├─────────────────────────────────────────────────────┤
│           Database (PostgreSQL)                      │
│  ┌──────────────────────────────────────────────┐   │
│  │  Tables: users, projects, tasks              │   │
│  │  Indexes: optimized queries                  │   │
│  │  Constraints: referential integrity          │   │
│  └──────────────────────────────────────────────┘   │
└─────────────────────────────────────────────────────┘
\end{verbatim}
\end{tcolorbox}

\section{Data Flow}

\begin{enumerate}
    \item User submits credentials on Login page
    \item Frontend sends POST request to /api/auth/login
    \item Backend validates credentials against database
    \item JWT token generated and returned to frontend
    \item Frontend stores token in localStorage
    \item Subsequent requests include token in Authorization header
    \item Middleware verifies token and attaches userId to request
    \item Routes use userId to fetch/create user-specific data
    \item Backend returns filtered data to frontend
    \item Frontend updates UI and displays results
\end{enumerate}

\chapter{Project File Structure}

\section{Frontend Structure}

\begin{verbatim}
frontend/
├── src/
│   ├── pages/
│   │   ├── LoginPage.jsx
│   │   ├── RegisterPage.jsx
│   │   └── Dashboard.jsx
│   ├── components/
│   │   └── Navbar.jsx
│   ├── App.jsx
│   ├── main.jsx
│   ├── index.css
│   └── App.css
├── package.json
├── vite.config.js
└── tailwind.config.js
\end{verbatim}

\section{Backend Structure}

\begin{verbatim}
backend/
├── src/
│   ├── routes/
│   │   ├── authRoutes.js
│   │   ├── dashboardRoutes.js
│   │   └── projectRoutes.js
│   ├── controllers/
│   │   └── authController.js
│   ├── middleware/
│   │   └── authMiddleware.js
│   ├── models/
│   │   ├── schema.sql
│   │   └── seed.sql
│   ├── db.js
│   └── server.js
├── .env
├── package.json
└── Dockerfile
\end{verbatim}

% ========== BACK MATTER ==========
\newpage
\thispagestyle{empty}
\vfill
\centering
\textit{End of Portfolio Document}

\end{document}
